\documentclass[11pt]{article}
\textwidth = 6.5in
\textheight = 8.75in
\hoffset=-.8in
\voffset=-.8in


\usepackage[pass]{geometry}
\usepackage{color}   % Necessary for colored links
\usepackage{xcolor}   % for 'violet' line numbers 
\usepackage[hypcap=true]{caption}       % point to top of figure rather than caption in \hyperref
%\usepackage{hyperref}
\usepackage[pagebackref]{hyperref}
\hypersetup{
    colorlinks=true,    %set true if you want colored links
    linkcolor=blue      %choose some color if you want links to stand out
}
\usepackage{lastpage}
\usepackage{fancyhdr}
\usepackage{sectsty}
\usepackage{amsmath}
\usepackage{listings}
\usepackage{marginnote}
\usepackage{enumitem}



\newcommand{\theListingFontFamily}{\ttfamily\footnotesize}

\lstnewenvironment{tty}
    {\lstset{language=sh, %
        numbers=left,numberstyle=\tiny,stepnumber=1,numbersep=8pt, %
        breaklines=true, %
        frame=single, %
        showtabs=false, %
        basicstyle=\theListingFontFamily, %
        showstringspaces=false, %
        tabsize=4, %
        showlines=true, %  show all blank lines
        basicstyle=\theListingFontFamily}}
    {}









\newcommand\MyTitle{CSCI 463 Assignment 4 -- RV32I Disassembler}

%\lstset{numbers=left, numberstyle=\small, stepnumber=1, numbersep=5pt, language={c++}}
\lstset{numbers=left, numberstyle=\small, stepnumber=1, numbersep=5pt, basicstyle=\footnotesize\ttfamily}

\sectionfont{\Large\sf\bfseries}
\subsectionfont{\large\sf\bfseries}
\subsubsectionfont{\large\sf\bfseries}

%%%%%%%%%%%%%%%%%%%%%%%%%%%%%%%%%%%%%%%%%%%%%%%%%%%%%%%%%%%%%%%%%%%%%%%%%%%%%%
%%%%%%%%%%%%%%%%%%%%%%%%%%%%%%%%%%%%%%%%%%%%%%%%%%%%%%%%%%%%%%%%%%%%%%%%%%%%%%
\pagestyle{fancy}
% supress normal headings and footters
\fancyhf{}
% remove the heading rule
%\renewcommand{\headrulewidth}{0pt}
\renewcommand{\headrulewidth}{1pt}
\renewcommand{\footrulewidth}{1pt}

\newcommand\lfText{{\sf\scriptsize Copyright \copyright\ 2020, 2021 John Winans. All Rights Reserved}\\
{\sf\scriptsize\FooterText}}

\lfoot{\lfText}
\rfoot{Page \thepage\ of \pageref*{LastPage}}
\lhead{\MyTitle}

% Sub-footer that shows the git version in the lfoot defined above
\ifdefined\GitFileName
    \newcommand{\FooterText}{\tt \GitFileName\\
\GitDescription}
\else
    \newcommand{\FooterText}{\emph{--UNKNOWN--}}
\fi

% force a footer (but no header) onto the first page as well
\fancypagestyle{plain}{%
\renewcommand{\headrulewidth}{0pt} %
\fancyhf{} % clear all header and footer fields
\lfoot{\lfText} %
\rfoot{Page \thepage\ of \pageref{LastPage}}
}

%%%%%%%%%%%%%%%%%%%%%%%%%%%%%%%%%%%%%%%%%%%%%%%%%%%%%%%%%%%%%%%%%%%%%%%%%%%%%%
%%%%%%%%%%%%%%%%%%%%%%%%%%%%%%%%%%%%%%%%%%%%%%%%%%%%%%%%%%%%%%%%%%%%%%%%%%%%%%
% For one-sided line numbers in the left margin
%\usepackage{lineno}
%\linenumbers
%\renewcommand\linenumberfont{\normalfont\tiny\sffamily\bfseries\color{violet}}
%\setlength\linenumbersep{6mm}


%%%%%%%%%%%%%%%%%%%%%%%%%%%%%%%%%%%%%%%%%%%%%%%%%%%%%%%%%%%%%%%%%%%%%%%%%%%%%%
%%%%%%%%%%%%%%%%%%%%%%%%%%%%%%%%%%%%%%%%%%%%%%%%%%%%%%%%%%%%%%%%%%%%%%%%%%%%%%


\setlength{\parindent}{0pt}
\setlength{\parskip}{.51em}

% do not number the sections
%\setcounter{secnumdepth}{0}


%%%%%%%%%%%%%%%%%%%%%%%%%%%%%%%%%%%%%%%%%%%%%%%%%%%%%%%%%%%%%%%%%%%%%%%%%%%%%%
%%%%%%%%%%%%%%%%%%%%%%%%%%%%%%%%%%%%%%%%%%%%%%%%%%%%%%%%%%%%%%%%%%%%%%%%%%%%%%
\begin{document}
%\pagenumbering{gobble}
 
\thispagestyle{plain}
\centerline{\huge\MyTitle}
\vspace{.2in}
\centerline{20 Points}

\begin{abstract}
\noindent In this assignment you will load RV32I binary executable file
into your memory simulator and decode/disassemble the instructions.

This is the second of a multi-part assignment involving the creation of a machine 
capable of executing programs compiled with g++.
The purpose of this assignment gain a thorough understanding of the RV32I 
machine instructions.
\end{abstract}

%%%%%%%%%%%%%%%%%%%%%%%%%%%%%%%%%%%%%%%%%%%%%%%%%%%%%%%%%%%%%%%%%%%%%%%%%
\section{Problem Description}

Disassemble an executable binary file by loading it into a simulated
memory of sufficient size and then decode each 32-bit instruction one-at-a-time
starting from address zero and continue through to the end of the simulated 
memory. 

%%%%%%%%%%%%%%%%%%%%%%%%%%%%%%%%%%%%%%%%%%%%%%%%%%%%%%%%%%%%%%%%%%%%%%%%%
\section{Files You Must Write}

You will write a C++ program suitable for execution on 
\verb@hopper.cs.niu.edu@ (or \verb@turing.cs.niu.edu@.)

Your source files {\em MUST} be named exactly as shown below or they will 
fail to compile and you will receive zero points for this assignment.

Create a directory named \verb@a4@ and place within it a copy of all the
the source files from Assignment 3 plus the \verb@rv32i_decode@  files defined below.

\begin{itemize}[leftmargin=1in]
\item[\tt main.cpp] Your \verb@main()@ and \verb@usage()@ function definitions will go here.
\item[\tt hex.h] The declarations of your hex formatting functions will go here.
\item[\tt hex.cpp] The definitions of your hex formatting functions will go here.
\item[\tt memory.h] The definition of your \verb@memory@ class will go here.
\item[\tt memory.cpp] The \verb@memory@ class member function definitions will go here.

\item [\tt rv32i\_decode.h] The definition of a class named \verb@rv32i_decode@ will go here.
\item [\tt rv32i\_decode.cpp] The definitions of any member functions of class \verb@rv32i_decode@ will go here.
\end{itemize}

%%%%%%%%%%%%%%%%%%%%%%%%%%%%%%%%%%%%%%%%%%%%%%%%%%%%%%%%%%%%%%%%%%%%%%%%%
\subsection{\tt main.cpp}

You must provide a \verb@main()@ function that is implemented as shown 
below plus a new utility function to disassemble the memory contents.  
This version of \verb@main()@ differs from the that in Assignment 3 by 
replacing the memory test function calls with the use of the new 
\verb@rv32i_decode@ class and the addition of a \verb@disassemble()@
helper function:

\begin{quote}
{\small
\begin{verbatim}
static void disassemble(const memory &mem)
{
    ...
}

int main(int argc, char **argv)
{
    uint32_t memory_limit = 0x100;      // default memory size = 256 bytes
    int opt;
    while ((opt = getopt(argc, argv, "m:")) != -1)
    {
        switch (opt)
        {
        case 'm':
            {
                std::istringstream iss(optarg);
                iss >> std::hex >> memory_limit;
            }
            break;
        default: /* '?' */
            usage();
        }
    }

    if (optind >= argc)
        usage();    // missing filename

    memory mem(memory_limit);

    if (!mem.load_file(argv[optind]))
        usage();

    disassemble(mem);
    mem.dump();

    return 0;
}
\end{verbatim}
}
\end{quote}

The \verb@disassemble()@ function is what will decode and print each instruction in a loop.

For each 32-bit word in the simulated memory:
\begin{itemize}
\item 
Call \verb@rv32i_decode::decode()@ to format/render one instruction mnemonic 
and return it as a \verb@std::string@.  

\item 
Print the memory address, instruction hex value, and the instruction mnemonic.
\end{itemize}

See the example output files below and on the course web page for details on 
how to align and space the output. 





%%%%%%%%%%%%%%%%%%%%%%%%%%%%%%%%%%%%%%%%%%%%%%%%%%%%%%%%%%%%%%%%%%%%%%%%%
\subsection{{\tt hex.h} and {\tt hex.cpp}}

Same as Assignment 3 plus the following additional methods:

\begin{itemize}
\item \verb@static std::string to_hex0x20(uint32_t i);@

This function must return a \verb@std::string@ beginning with \verb@0x@, followed by
the 5 hex digits representing the 20 least significant bits of the \verb@i@ argument.  
Implement this method in 
a similar fashion as \verb@hex::to_hex8()@ but without the \verb@static_cast@.

This new method will be used when formatting the \verb@lui@ and \verb@auipc@ instructions.

\item \verb@static std::string to_hex0x12(uint32_t i);@

This function must return a \verb@std::string@ beginning with \verb@0x@, followed by
the 3 hex digits representing the 12 least significant bits of the \verb@i@ argument.
See \verb@to_hex0x20()@.

This new method will be used when formatting the \verb@csrrX()@ instructions.


\end{itemize}




%%%%%%%%%%%%%%%%%%%%%%%%%%%%%%%%%%%%%%%%%%%%%%%%%%%%%%%%%%%%%%%%%%%%%%%%%
\subsection{{\tt memory.h} and {\tt memory.cpp}}

Same as for Assignment 3. 





%%%%%%%%%%%%%%%%%%%%%%%%%%%%%%%%%%%%%%%%%%%%%%%%%%%%%%%%%%%%%%%%%%%%%%%%%
\subsection{{\tt rv32i\_decode.h} and {\tt rv32i\_decode.cpp}}

Your \verb@rv32i_decode@ class must, at least, include the members 
discussed below plus appropriate documentation.

Failure to follow this design will cause significant problems with future 
assignments implemented by extending this assignment.

Note that {\em all} of the members of this class are \verb@static@ and 
{\em none} will actually ``print'' anything to the screen.

The \verb@rv32i_decode@ class is a subclass of the \verb@hex@ class.  
Therefore it will inherit all the hex formatting methods.



%%%%%%%%%%%%
\subsubsection{{\tt rv32i\_decode} Member Constants}

\begin{itemize}

\item \verb@static constexpr uint32_t XLEN = 32;@

\verb@XLEN@ represents the number of bits in RV32I CPU registers.

\item You will find it useful to define a number of constants that 
represent the binary values of the opcode, funct3, and funct7 fields 
of each of the instructions as well as some useful widths for 
formatting the returned string values.  For example:

\begin{quote}
\small
\begin{verbatim}
static constexpr int mnemonic_width     = 8;
static constexpr uint32_t opcode_lui    = 0b0110111;
static constexpr uint32_t opcode_auipc  = 0b0010111;
...
static constexpr uint32_t funct3_beq    = 0b000;
static constexpr uint32_t funct3_lb     = 0b000;
static constexpr uint32_t funct3_sll    = 0b001;
...
\end{verbatim}
\end{quote}

Note that these opcode constants are expressed with {\em binary} literals.  
This notation was added in \verb@C++14@.  
(Make sure you use the proper compiler flags.)

A file with the above and other useful constants will be made available
on the course web page.
 
\end{itemize}






%%%%%%%%%%%%
\subsubsection{{\tt rv32i\_decode} Member Functions}

\begin{itemize}

\item \verb@static std::string decode(uint32_t addr, uint32_t insn);@

It is the purpose of this function to return a \verb@std::string@ containing 
the disassembled instruction text. 

It must be capable of handling any possible 32-bit \verb@insn@ value (regardless of
it representing a useful/legal instruction or not.)

Note that the memory address parameter \verb@addr@, from which the
instruction has been fetched, is passed to \verb@decode()@ so that it 
may be used to calculate the \verb@PC@-relative target address 
shown in the J-type and B-type instructions.

You can implement this software decoder by using a \verb@switch@ statement
with the value of the \verb@opcode@ field by extracting it from \verb@insn@
by calling \verb@get_opcode(insn)@.

For each \verb@case@, call a function to format/render the instruction 
and its arguments (when the opcode is specific enough to do so such as 
for the \verb@lui@ instruction) or use a sub-\verb@switch@ statement to 
further decode any additionally required fields (such as \verb@funct3@ 
and/or \verb@funct7@) and then call a function to format the instruction 
as shown in \autoref{hints}.


Note that, most of the instructions are in groups such that the disassembled 
instruction {\em format} is the same.  For example, all of the {\em store} 
instructions differ only in their mnemonic.

Take advantage of this by factoring your \verb@std::string@ formatting logic
to minimize the amount of replicated code.  To do this, add formatting 
helper-functions like these to the \verb@rv32i_decode@ class as needed:

\begin{quote}
\small
\begin{verbatim}
static std::string render_illegal_insn();
static std::string render_auipc(uint32_t insn);
static std::string render_jal(uint32_t addr, uint32_t insn);
static std::string render_btype(uint32_t addr, uint32_t insn, const char *mnemonic);
static std::string render_itype_load(uint32_t insn, const char *mnemonic);
static std::string render_itype_alu(uint32_t insn, const char *mnemonic, int32_t imm_i);
static std::string render_stype(uint32_t insn, const char *mnemonic);
...
\end{verbatim}
\end{quote}

The simplest of these is the renderer for invalid instructions:

{\small
\begin{quote}
\begin{verbatim}
std::string rv32i_decode::render_illegal_insn()
{
    return "ERROR: UNIMPLEMENTED INSTRUCTION";
}
\end{verbatim}
\end{quote}
}

See the \verb@render_lui()@ example in \autoref{hints} for a more interesting example.

These above \verb@render_X()@ functions should also use helper-functions like 
those below to eliminate messy replicated code:

\begin{itemize}
\item \verb@static std::string render_reg(int r);@

Render, into a \verb@std::string@, the name of the register with the given 
number \verb@r@.

For example, \verb@render_reg(23)@ would return a string with the value:
\verb@x23@

Calling this member is the one and only way that any code may format the name of 
a register into a string (for subsequent printing by the program.)

\item \verb@static std::string render_base_disp(uint32_t register, int32_t imm);@

Use this to render, into a \verb@std::string@, the operands of the 
form \verb@imm(register)@ for those instructions that have such
an addressing mode.

Note that, in this case, the \verb@imm@ value is printed in decimal.
The register should be rendered into a string by \verb@render_reg()@.

Calling this member is the one and only way that any code may format 
an \verb@imm(register)@ operand in your application.


\item \verb@static std::string render_mnemonic(const std::string &mnemonic);@

Render, into a \verb@std::string@, the given instruction mnemonic with the proper
space padding on right side to make it \verb@mnemonic_width@ characters long.

By using this to make all mnemonics a uniform width when printing instructions
that have operands, aligning the operands will be much easier.

Note that \verb@ecall@ and \verb@ebreak@ are the {\em only} instructions that 
do {\em not} have operands and that must therefore {\em not} be rendered with
padding on the right.

\end{itemize}


You should feel free to alter or add more helper-functions for rendering that you find useful.


\end{itemize}


The following helper-functions for extracting the fields from a 32-bit instruction 
will make the instruction decoding logic easier to write, debug, and understand:



\begin{itemize}

\item \verb@static uint32_t get_opcode(uint32_t insn);@

Extract and return the \verb@opcode@ field from the given instruction.
A suitable implementation is:

\begin{quote}
\verb@return (insn & 0x0000007f);@
\end{quote}

\item \verb@static uint32_t get_rd(uint32_t insn);@

Extract and return the \verb@rd@ field from the given instruction as 
an integer value from 0 to 31.

\item \verb@static uint32_t get_rs1(uint32_t insn);@

Extract and return the \verb@rs1@ field from the given instruction 
as an integer value from 0 to 31.

\item \verb@static uint32_t get_rs2(uint32_t insn);@

Extract and return the \verb@rs2@ field from the given instruction
an integer value from 0 to 31.

\item \verb@static uint32_t get_funct3(uint32_t insn);@

Extract and return the \verb@funct3@ field from the given instruction
as an integer value from 0 to 7.

\item \verb@static uint32_t get_funct7(uint32_t insn);@

Extract and return the \verb@funct7@ field from the given instruction
as an integer value from \verb@0x00@ to \verb@0x7f@.

\item \verb@static int32_t get_imm_i(uint32_t insn);@

Extract and return the \verb@imm_i@ field from the given instruction
as a 32-bit signed integer as shown in RVALP.

\item \verb@static int32_t get_imm_u(uint32_t insn);@

Extract and return the \verb@imm_u@ field from the given instruction
as a 32-bit signed integer as shown in RVALP.

\item \verb@static int32_t get_imm_b(uint32_t insn);@

Extract and return the \verb@imm_b@ field from the given instruction
as a 32-bit signed integer as shown in RVALP.

\item \verb@static int32_t get_imm_s(uint32_t insn);@

Extract and return the \verb@imm_s@ field from the given instruction.
A suitable implementation is:

\begin{quote}
{\small
\begin{verbatim}
int32_t rv32i_decode::get_imm_s(uint32_t insn)
{
    int32_t imm_s = (insn & 0xfe000000) >> (25-5);
    imm_s |= (insn & 0x00000f80) >> (7-0);

    if (insn & 0x80000000)
        imm_s |= 0xfffff000;    // sign-extend the left

    return imm_s;
}
\end{verbatim}
}
\end{quote}

Note that the shifting distances in the above example have been specified as a
calculation using the values
shown in the castellated diagrams in RVALP.  Showing the shift distance values 
such as: \verb@25-5@ is more clear than simply \verb@20@ because it a) illustrates 
your intent and b) prevents a silly subtraction mistake from creating a problem.

\item \verb@static int32_t get_imm_j(uint32_t insn);@

Extract and return the \verb@imm_j@ field from the given instruction
as a 32-bit signed integer as shown in RVALP.

\end{itemize}


%%%%%%%%%%%%%%%%%%%%%%%%%%%%%%%%%%%%%%%%%%%%%%%%%%%%%%%%%%%%%%%%%%%%%%%%%
\section{Input}

Your program will accept the same command-line arguments as Assignment 3.

The optional \verb@[-m hex-mem-size]@ argument is a hex number representing 
the amount of memory to simulate. When not given, the default value must be 
\verb@0x100@.

The last argument is the name of a file to load into the simulated memory. 
In this assignment, this will be the name of a \verb@RV32I@ binary executable 
program.

You will be provided with a number of suitable executable test programs and
associated output files.

%%%%%%%%%%%%%%%%%%%%%%%%%%%%%%%%%%%%%%%%%%%%%%%%%%%%%%%%%%%%%%%%%%%%%%%%%
\section{Output}

Your program's output will be a disassembly of the executable binary file
followed by a dump of the simulated memory.


The disassembled instructions are rendered with a mix of decimal and hex values:

\begin{itemize}
\item The far left column is the 32-bit address printed in hex.
\item The second column is the 32-bit hex full-word representation of the instruction 
that was fetched from your memory at the address in column 1.
\item The third column is the instruction mnemonic.
\item The presence and format of the fourth column depends on the type of the instruction.
\begin{itemize}
\item Register numbers are printed as an 'x' followed by the decimal number of the register (0-31)
	as in \verb@x12@, \verb@x0@, and \verb@x31@.
\item Hexadecimal literals are printed with a leading \verb@0x@ as in \verb@0xabcde@.
\item Decimal literals are printed with the first character being an optional
negative sign `-` followed by one or more digits (0-9) as in \verb@1234@ and \verb@-58@.
\end{itemize}
\end{itemize}

It {\em seems} unfortunate that the letter \verb@x@ was chosen for the register 
indicator since it is so close to (and could be confused with) the indicator of 
hexadecimal numbers \verb@0x@.  Be careful!

See below for a list of every instruction and the proper disassembly format. 
(A copy of this output and others are available on the course web site.)
Your program must match the reference output precisely (including whitespace.)

{\footnotesize
\begin{verbatim}
winans@hopper:~$ ./rv32i -mc0 testdata/allinsns4.bin
00000000: abcde237  lui     x4,0xabcde
00000004: abcde217  auipc   x4,0xabcde
00000008: 004000ef  jal     x1,0x0000000c
0000000c: 00808267  jalr    x4,8(x1)
00000010: ff808267  jalr    x4,-8(x1)
00000014: 00000263  beq     x0,x0,0x00000018
00000018: fe001ee3  bne     x0,x0,0x00000014
0000001c: 00004263  blt     x0,x0,0x00000020
00000020: 00005263  bge     x0,x0,0x00000024
00000024: 00006263  bltu    x0,x0,0x00000028
00000028: 00007263  bgeu    x0,x0,0x0000002c
0000002c: 4d200203  lb      x4,1234(x0)
00000030: b2e01203  lh      x4,-1234(x0)
00000034: 4d202203  lw      x4,1234(x0)
00000038: 80004203  lbu     x4,-2048(x0)
0000003c: 00005203  lhu     x4,0(x0)
00000040: 4c400923  sb      x4,1234(x0)
00000044: 4c401923  sh      x4,1234(x0)
00000048: 4c402923  sw      x4,1234(x0)
0000004c: 4d228213  addi    x4,x5,1234
00000050: fff3a313  slti    x6,x7,-1
00000054: 8304b413  sltiu   x8,x9,-2000
00000058: 0015c513  xori    x10,x11,1
0000005c: 7d06e613  ori     x12,x13,2000
00000060: 4d27f713  andi    x14,x15,1234
00000064: 00c89813  slli    x16,x17,12
00000068: 00c9d913  srli    x18,x19,12
0000006c: 40cada13  srai    x20,x21,12
00000070: 018b8b33  add     x22,x23,x24
00000074: 41bd0cb3  sub     x25,x26,x27
00000078: 01ee9e33  sll     x28,x29,x30
0000007c: 0020afb3  slt     x31,x1,x2
00000080: 0020b233  sltu    x4,x1,x2
00000084: 0020c233  xor     x4,x1,x2
00000088: 0042d1b3  srl     x3,x5,x4
0000008c: 4042d1b3  sra     x3,x5,x4
00000090: 0020e233  or      x4,x1,x2
00000094: 0020f233  and     x4,x1,x2
00000098: 000b1a73  csrrw   x20,0x000,x22
0000009c: 001cabf3  csrrs   x23,0x001,x25
000000a0: 002e3d73  csrrc   x26,0x002,x28
000000a4: fff55ef3  csrrwi  x29,0xfff,10
000000a8: 10066f73  csrrsi  x30,0x100,12
000000ac: 00177ff3  csrrci  x31,0x001,14
000000b0: 001ffff3  csrrci  x31,0x001,31
000000b4: 00100073  ebreak
000000b8: 00000073  ecall
000000bc: a5a5a5a5  ERROR: UNIMPLEMENTED INSTRUCTION
00000000: 37 e2 cd ab 17 e2 cd ab  ef 00 40 00 67 82 80 00 *7.........@.g...*
00000010: 67 82 80 ff 63 02 00 00  e3 1e 00 fe 63 42 00 00 *g...c.......cB..*
00000020: 63 52 00 00 63 62 00 00  63 72 00 00 03 02 20 4d *cR..cb..cr.... M*
00000030: 03 12 e0 b2 03 22 20 4d  03 42 00 80 03 52 00 00 *....." M.B...R..*
00000040: 23 09 40 4c 23 19 40 4c  23 29 40 4c 13 82 22 4d *#.@L#.@L#)@L.."M*
00000050: 13 a3 f3 ff 13 b4 04 83  13 c5 15 00 13 e6 06 7d *...............}*
00000060: 13 f7 27 4d 13 98 c8 00  13 d9 c9 00 13 da ca 40 *..'M...........@*
00000070: 33 8b 8b 01 b3 0c bd 41  33 9e ee 01 b3 af 20 00 *3......A3..... .*
00000080: 33 b2 20 00 33 c2 20 00  b3 d1 42 00 b3 d1 42 40 *3. .3. ...B...B@*
00000090: 33 e2 20 00 33 f2 20 00  73 1a 0b 00 f3 ab 1c 00 *3. .3. .s.......*
000000a0: 73 3d 2e 00 f3 5e f5 ff  73 6f 06 10 f3 7f 17 00 *s=...^..so......*
000000b0: f3 ff 1f 00 73 00 10 00  73 00 00 00 a5 a5 a5 a5 *....s...s.......*
\end{verbatim}
}



%%%%%%%%%%%%%%%%%%%%%%%%%%%%%%%%%%%%%%%%%%%%%%%%%%%%%%%%%%%%%%%%%%%%%%%%%
\section{How To Hand In Your Program}

When you are ready to turn in your assignment, make sure that the only files in your
\verb@a4@ directory is/are the source files defined and discussed above.
Then, in the parent of your \verb@a4@ directory, use the \verb@mailprog.463@ 
command to send the contents of the files in your \verb@a4@ project directory 
in the same manner as we have used in the past.


%%%%%%%%%%%%%%%%%%%%%%%%%%%%%%%%%%%%%%%%%%%%%%%%%%%%%%%%%%%%%%%%%%%%%%%%%
\section{Grading}

The grade you receive on this programming assignment will be scored according to the
syllabus and its ability to compile and execute on the Computer Science Department's 
computer.

{\em It is your responsibility to test your program thoroughly.} 

When we grade your assignment, we will compile it on \verb@hopper.cs.niu.edu@ 
using these exact commands:

{\small
\begin{verbatim}
g++ -g -ansi -pedantic -Wall -Werror -std=c++14 -c -o main.o main.cpp
g++ -g -ansi -pedantic -Wall -Werror -std=c++14 -c -o rv32i_decode.o rv32i_decode.cpp
g++ -g -ansi -pedantic -Wall -Werror -std=c++14 -c -o memory.o memory.cpp
g++ -g -ansi -pedantic -Wall -Werror -std=c++14 -c -o hex.o hex.cpp
g++ -g -ansi -pedantic -Wall -Werror -std=c++14 -o rv32i main.o rv32i_decode.o memory.o hex.o  
\end{verbatim}
}

Your program will then be run multiple times using different memory sizes and
binary test files.

%%%%%%%%%%%%%%%%%%%%%%%%%%%%%%%%%%%%%%%%%%%%%%%%%%%%%%%%%%%%%%%%%%%%%%%%%
\section{Hints}
\label{hints}

While this is the second part of a multi-part assignment, it too should be 
written in parts!

\begin{itemize}
\item
Start by updating \verb@main.cpp@ and stub in the \verb@disassemble()@
method such that it does nothing but return so you can just get your new framework to
compile and verify that is OK by running it and see that \verb@mem.load_file()@
and \verb@mem.dump()@ still work OK.

\item
Implement the \verb@disassemble()@ loop to just print the \verb@pc@ and 
fetched instruction-word values.


\item
Implement the \verb@render_illegal_insn()@ helper-function and then call it from
a first draft stub version of \verb@rv32i_decode::decode()@ that treats all the instructions 
as illegal. It might look something like this:

{\small
\begin{verbatim}
std::string rv32i_decode::decode(uint32_t addr, uint32_t insn)
{
    switch(get_opcode(insn))
    {
    default:   return render_illegal_insn();
    }
    assert(0 && "unrecognized opcode"); // It should be impossible to ever get here!
}
\end{verbatim}
}

\ldots and your output would look something like this:

{\small
\begin{verbatim}
winans@ux410ua~$ ./rv32i -m48 testdata/tinyprog.bin
00000000: 00000137  ERROR: UNIMPLEMENTED INSTRUCTION
00000004: 00001137  ERROR: UNIMPLEMENTED INSTRUCTION
00000008: 80000137  ERROR: UNIMPLEMENTED INSTRUCTION
...
00000000: 37 01 00 00 37 11 00 00  37 01 00 80 37 01 00 40 *7...7...7...7..@*
00000010: 37 f1 ff 4f 37 f1 ff ff  37 f0 ff ff 17 01 00 00 *7..O7...7.......*
...
\end{verbatim}
}
 
Note that it is OK to have a declaration of a method in your 
\verb@rv32i_decode@ class without actually defining it as long as you don't 
call it.

Note the use of an assertion to make sure that execution can never fall off the end
of the switch statement without stopping the program.  Given the style of this
particular switch statement design that includes a default case with a \verb@return@
and, as will be shown below, any cases added as the solution is evolved will
also include a return statement, execution of the code can never run past the
end of the switch statement.  If it does then it means that the code in the switch 
statement is broken and the failing assertion will terminate the program and let us know.

\item
Add each instruction opcode one at-a-time starting with \verb@lui@ like this:

{\small
\begin{verbatim}
std::string rv32i_decode::decode(uint32_t addr, uint32_t insn)
{
    switch(get_opcode(insn))
    {
    default:            return render_illegal_insn();
    case opcode_lui:    return render_lui(insn);
    }
    assert(0 && "unrecognized opcode"); // It should be impossible to ever get here!
}

std::string rv32i_decode::render_lui(uint32_t insn)
{
    uint32_t rd = get_rd(insn);
    int32_t imm_u = get_imm_u(insn);

    std::ostringstream os;
    os << render_mnemonic("lui") << render_reg(rd) << "," 
       << to_hex0x20((imm_u >> 12)&0x0fffff);
    return os.str();
}
\end{verbatim}
}

The output would then be:

{\small
\begin{verbatim}
./rv32i -m48 testdata/tinyprog.bin
00000000: 00000137  lui     x2,0x00000
00000004: 00001137  lui     x2,0x00001
00000008: 80000137  lui     x2,0x80000
...
\end{verbatim}
}


\item
When you write the \verb@get_imm_X()@ and \verb@get_functX()@ 
methods {\em test them with a series of useful hard-coded values to make sure you 
have all the bits repositioned and working correctly. It can save you a
HUGE amount of debugging time if you exhaustively test them with many bit 
patterns before using them.}

Extract the field data in these methods by using using a combination of 
the bitwise {\em and} \verb@&@, {\em or} \verb@|@, 
and {\em shift} operators: \verb@>>@ and \verb@<<@
as seen in the \verb@get_imm_s()@ example code seen earlier.

\item
Implement and test one opcode at-a-time using 
the reference card and the diagrams showing how the instruction fields
are decoded in ``RISC-V Assembly Language Programming'' (AKA RVALP.) 
The latest version is located here: 
\url{https://github.com/johnwinans/rvalp/releases/} (Click on the 
``Assets'' menu for the latest/top shown release and open ``book.pdf.'')


\item
Do all the the easy ones first such as the U-type and R-type instructions 
that appear at the beginning of the \verb@tinyprog.bin@ example file. 

\item
When implementing instructions that share the same rendering code, use one
helper for all of them.  For example, the \verb@render_btype()@ can handle 
the rendering any of the branch instructions and a \verb@render_itype_alu()@ 
helper-function can handle the rendering any of I-type instructions that 
perform ALU operations as can be seen below:

{\small
\begin{verbatim}
std::string rv32i_decode::decode(uint32_t addr, uint32_t insn)
{
    switch(get_opcode(insn))
    {
    default:                    return render_illegal_insn();
    case opcode_lui:            return render_lui(insn);
    ....
    case opcode_btype:
        switch (funct3)
        {
        default:                return render_illegal_insn();
        case funct3_beq:        return render_btype(addr, insn, "beq"); 
        case funct3_bne:        return render_btype(addr, insn, "bne"); 
        ...
        }
        assert(0 && "unrecognized funct3");  // impossible
    case opcode_alu_imm:
        switch (funct3)
        {
        default:                return render_illegal_insn();
        case funct3_add:        return render_itype_alu(insn, "addi", get_imm_i(insn));
        ...
        case funct3_sll:        return render_itype_alu(insn, "slli", get_imm_i(insn)%XLEN);
        case funct3_srx: 
            switch(funct7)
            {
            default:            return render_illegal_insn(); 
            case funct7_sra:    return render_itype_alu(insn, "srai", get_imm_i(insn)%XLEN);
            case funct7_srl:    return render_itype_alu(insn, "srli", get_imm_i(insn)%XLEN);
            }
            assert(0 && "unrecognized funct7");  // impossible
        }
        assert(0 && "unrecognized funct3");  // impossible
    case ...
    ...
    }
    assert(0 && "unrecognized opcode"); // It should be impossible to ever get here!
}
\end{verbatim}
}

{\em Note that the \verb@shamt@ value passed to the \verb@render_itype_alu()@
helper-function for the shift-immediate instructions is the same thing as 
the \verb@imm_i@ value if we ignore all but the least-significant XLEN 
bits\ldots\ which is the remainder if we divide the \verb@imm_i@ by XLEN.}


\end{itemize}


\end{document}
